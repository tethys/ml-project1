\documentclass{article} % For LaTeX2e
% We will use NIPS submission format
\usepackage{nips13submit_e,times}
% for hyperlinks
\usepackage{hyperref}
\usepackage{url}
% For figures
\usepackage{graphicx} 
% math packages
\usepackage{amsmath}
\usepackage{amsfonts}
\usepackage{amsopn}
\usepackage{ifthen}
\usepackage{natbib}
\usepackage{caption}
\usepackage{subcaption}

\usepackage{floatrow}
% Table float box with bottom caption, box width adjusted to content
\newfloatcommand{capbtabbox}{table}[][\FBwidth]

\title{Project-I by Group Rome}

\author{
Gocken Cimen\\
EPFL \\
\texttt{gcimen@epfl.ch} \\
\And
Viviana Petrescu\\
EPFL \\
\texttt{viviana.petrescu@epfl.ch} \\
\And
Angelopoulos Vasileios \\
EPFL \\
\texttt{vasileios.angelopoulos@epfl.ch} \\
}

% The \author macro works with any number of authors. There are two commands
% used to separate the names and addresses of multiple authors: \And and \AND.
%
% Using \And between authors leaves it to \LaTeX{} to determine where to break
% the lines. Using \AND forces a linebreak at that point. So, if \LaTeX{}
% puts 3 of 4 authors names on the first line, and the last on the second
% line, try using \AND instead of \And before the third author name.

\nipsfinalcopy 

\begin{document}

\maketitle

\begin{abstract}
We present our on two tasks, classification and regression on data that is not known. We process the data, investigate baseline methods and present our results here.
For the regression task, out best model was <> and for classification was <>.
\end{abstract}

\section{Regression}
\subsection{Data Description}
The training data $Xtrain$ contains 1400 observations, each 43 dimensional. One training sample has 36 real valued features and 7 categorical features. Our task is to predict the values for unseen test data, consisting in 600 samples. We measure the accuracy of our estimation using $RMSE$. 

\subsection{Data visualization and cleaning}
Initially, our data was not centered, as seen in $Fig.$ \ref{fig:dist_regression}.
We changed the categorical features into dummy variables, leading to a new vector of size 56. $Xtrain$ was then normalized to have 0 mean and standard deviation 1 (except the dummy variables). We applied the same operations to the test data $Xtest$ on which we will report the results.

 We plotted the correlation of every feature with respect to the output $ytrain$ and the scatter plots did not look random. We concluded the features explain the output, but we could not tell if one of them is insignificant.

The predicted values were real values in $[1000,7000]$ and seemed to be grouped into two blobs. We believed initially the smaller blob represented outliers, but since it contained almost $10\%$ of the data we decided not to ignore it.

After looking at every feature individually, we noticed that feature 36 offers a clear separation of the two blobs (see $Fig.$ \ref{fig:feature36}). We therefore chose to fit two models, one in which feature 36 has values $>1.4$ (after normalization) and one in which it is smaller, corresponding to smaller predicted values. 


\begin{figure}[ht]
  \begin{subfigure}[b]{0.45\textwidth}
   \includegraphics[width=\textwidth]{figures/distribution_regression_crop.pdf}
    \caption{Mean and standard deviation for the first 36 real valued variables of Xtrain. The input is not normalised and feature 36 is the only negative one.}
    \label{fig:dist_regression}
  \end{subfigure}
  \begin{subfigure}[b]{0.45\textwidth}
    \includegraphics[width=\textwidth]{figures/feature36_crop.pdf}
    \caption{Feature 36 versus output values. X = 1.4 (black line) and y = 4900 (red line)  provide a good separation of the two blobs.}
    \label{fig:feature36}
  \end{subfigure}
  \caption{• Data visualization. }
\end{figure}

We visually observed some linear correlations between certain features such as feature 2 and 24, 13 and 16, 17 and 20, but we decided to keep them since we did not have time to experiment with their removal or to test their signifcance. This is corroborated by the fact that $Xtrain$ is rank deficient, it has 57 columns after the use of dummy variables but rank 50.

If the input is normally distributed with mean 0 and std 1, then
 $99.99\%$ of the samples appear between the values -3.891 and 3.891. We therefore remove any points that are outside this interval, considering them outliers.

\subsection{Ridge regression Baseline methods}
 

\subsection{Feature transformations}
We tried polynomial and exponential transformations of the features.
The exponential transformation did not prove to work well and we focused on polynomial regression.
We only transformed the first 36 variables and kept the categorical variables as they are since  polynomial transformation of categorical variables does not have an intuitive interpretation. 

The degree of the polynomial was varied from 1 to 10 for both blob-models. In $Fig$\ref{fig:degree_blob1},\ref{fig:degre_blob2} we plotted the mean training and validation $RMSE$ for the degree varying only between 2 - 10 and 2 - 6 respectively for readability (for the other values the RMSE was too big). We will refer to the scatter points grouped in a round shape as the first blob or big blob and the the other points as the small blob.

Using 5-fold cross validation with $\lambda = 1e-5, \alpha = 0.1$ we
notice that a polynomial of degree 3 is a good fit for the first model. Increasing it further leads to overfitting, since the training errors becomes very small and the validation error increases. For the second model, using  $\lambda = 0.001, \alpha = 0.1$ we obtain a best fit for a polynomial of degree 2.

\begin{figure}[h]
  \begin{subfigure}[b]{0.45\textwidth}
   \includegraphics[width=\textwidth]{figures/degree_polynomial_blob1_crop.pdf}
    \caption{First Blob. Mean RMSE for training set \newline (blue curve) and testing (red curve) versus poly-\newline nomial degree. The errors were computed using\newline 5-fold cross-validation.}
    \label{fig:degree_blob1}
  \end{subfigure}
  \begin{subfigure}[b]{0.45\textwidth}
    \includegraphics[width=\textwidth]{figures/degree_polynomial_blob2_crop.pdf}
    \caption{Second Blob. Mean RMSE for training set (blue curve) and testing (red curve) versus polynomial degree. The errors were computed using 5-fold cross-validation.}
    \label{fig:degre_blob2}
  \end{subfigure}
  \caption{• Selection of polynomial degree}
\end{figure}

We increased lambda to prevent overfitting, since we have 10 times less samples for the second model. However, we still noticed that our RMSE for the validation set  was  bigger for the second blob. This might be due to the fact that we have a very small set for both training and testing of the second model.

Both polynomial regression models significantly outperform the normal ridge regression, which gave a RMSE > 100.

\begin{center}
  \begin{tabular}{ |l | c | r| }
    \hline
     & RMSE train & RMSE test \\ \hline
    Blob1 model & 5.37 ($\pm$ 0.044) & 6.25 ($\pm$ 0.4) \\ \hline
    Blob2 model & 7.16 ($\pm$ 1.85) & 58.10 ($\pm$ 16.2) \\
    \hline
  \end{tabular}
  	\label{table:feat_transform}
    \captionof{table}{Estimated Train and Test RMSE for the two blob models.}
\end{center}

Our wrong assumption was that in our training set we had a clear separation between the two models using the value of feature 36. The problem is that there are some samples with feature 36 > -13 whose predicted value should be estimated using model 1 and some which sould use model 2. This can also be seen in $Fig.$\ref{fig:feature36}.
 Because of this we expect our error on new data to be big for samples that have feature36 in the range [-12,-14].
To test this, we tried splitting $Xtrain$ into $20\%$ for testing and $80\%$ for training and validation. We applied the same data processing, model parameters as above and the RMSE per model proved to be similar with the ones in Table\ref{table:feat_transform} and as expected out test error got much worse. The total RMSE for our $20\%$ unseen test data
was   295.78 (RMSE= 73.5 for model 1, RMSE = 887.24 for model2).
If we remove all the samples in the test data with feature36 in the interval [-12,-14] we obtain a total RMSE=68.6223 (RMSE= 71.6 for model 1, RMSE = 82.77 for model2)
   

Our model is similar with cite book here and suffers from the same problems of the multiple regression problem as stated there.
Since we do not enforce continuiting at the X36 feature, our model will generate high errors for samples that have the value of feature36 near the threshold value. This happens because feature 36 does not offer a clear separation of the data. We could have tried to enforce smoothnes at the thresholded value. We believe we have two models that work well separately but perform bad at the border. We removed all samples between -12,-14 and indeed our RMSE got worse, showing that the worst error comes from that part of the model.


\section{Classification}


\subsection{Logistic Regression}

We applied Logistic regression and Penalized logistic Regression using gradient descent. Best results are obtained on the train accuracy of $97.14\%$ with value 10.0 for $\alpha$. We obtained the optimal $\alpha$ value for logistic regression classifier using cross validation technique where we split $50\%$ of the data for validation.

In penalized logistic regression, the choice of penalty factor ($\lambda$) is crucial. Therefore, we used cross validation procedure to estimate $\lambda$ when we divide $50\%$ of the data as test data and $50\%$ of the data as training data. The value of $\lambda$ varies from $10^{-2}$ to $10^3$ with 400 points in between. As can be seen in Figure \ref{fig:Lambda_pLr}, for smaller $\lambda$ values, training error is much lower than the test error while the test error gets as high as the training error  as the $\lambda$ increases which is a sign of under-fitting. 

For smaller $\lambda$ values, the training and test error estimated in penalized logistic regression shows very similar behavior in the training and test error estimated in logistic regression. The same $\alpha$ value that is obtained for the logistic regression method also gives the best accuracy for penalized logistic regression method.

\begin{figure}[h]
  \begin{subfigure}[b]{0.5\textwidth}
   \includegraphics[clip, trim=4cm 9cm 3cm 10cm, width=\textwidth]{figures/Lambda_pLG.pdf}
    \caption{Train and test error assessment with penalty factor ($\lambda$) for penalized logistic regression
    for 50-50 split.}
    \label{fig:Lambda_pLr}
  \end{subfigure}
  \hfill
  \begin{subfigure}[b]{0.45\textwidth}
    \includegraphics[clip, trim=4cm 10cm 3cm 10cm, width=\textwidth]{figures/comparison_LR_pLR.pdf}
    \caption{Comparison of logistic regression and penalized logistic regression based on validation accuracy}
    \label{fig:comp_LR_pLR}
  \end{subfigure}
\end{figure}

Best results are obtained on the test accuracy of $97.318\%$ with the values 0.1 for ($\lambda$) and 10.0 for ($\alpha$).
\ref{fig:comp_LR_pLR} shows the comparison of the logistic regression and penalized logistic regression based on the classification accuracy percentage over the test data. The improvements that is captured with the penalized logistic regression is little but significant.  

\begin{table}[h!]
\begin{center}
    \begin{tabular}{ | l | l | l | p{5cm} |}
    \hline
    Method & RMSE & 0-1-Loss & Log-Loss \\ \hline
    Logistic Regression & 0.147896 & 0.020743 & 105.151844 \\ \hline
    Pen Logistic Regression & 0.122288 & 0.014693 & 75.380959 \\ \hline
    \end{tabular}
\end{center}
\caption{Some prediction error estimates for the test data}
\label{table:test_errors}
\end{table}
 
The table \ref{table:test_errors} shows error measurements of the test data with $\alpha=10.0$ and $\lambda=1.0$ using $RMSE$, $0-1 loss$ and $log-loss$. Increasing the value of $\alpha$ decreases all error estimations until a maximum $\lambda=10.0$ value and we obtained the best prediction accuracy with this value for the $\lambda$.


\section{Summary}


\subsubsection*{Acknowledgments}

\subsubsection*{References}

\end{document}
